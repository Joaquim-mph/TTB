% Template:     Presentación LaTeX
% Documento:    Archivo principal
% Versión:      2.1.4 (12/01/2023)
% Codificación: UTF-8
%
% Autor: Pablo Pizarro R.
%        pablo@ppizarror.com
%
% Manual template: [https://latex.ppizarror.com/presentacion]
% Licencia MIT:    [https://opensource.org/licenses/MIT]

% CREACIÓN DEL DOCUMENTO
\pdfminorversion=7
\documentclass[
	spanish, % Idioma: spanish, english, etc.
	aspectratio=43, % 1610, 169, 149, 54, 43, 32
	hyperref={pdfencoding=auto,psdextra},
	xcolor={dvipsnames,table,usenames}
]{beamer}

% INFORMACIÓN DEL DOCUMENTO
\def\documenttitle {Midiendo tasa de formación estelar en análogos locales para galaxias de alto redshift}
\def\documentsubtitle {TTB}
\def\documentsubject {Mediante lineas de emisión en el infrarrojo cercano}

\def\documentauthor {Joaquín López Cortés}
\def\coursename {Trabajo Tutorial Básico}
\def\coursecode {AS-4901}

\def\universityname {Universidad de Chile}
\def\universityfaculty {Facultad de Ciencias Físicas y Matemáticas}
\def\universitydepartment {Departamento de Astronomía}
\def\universitydepartmentimage {departamentos/das}
\def\universitylocation {Santiago de Chile}

% CONFIGURACIÓN DATOS BEAMER
\title[\documentsubtitle]{\documenttitle}
\subtitle{\documentsubject}
\author[\documentauthor]{
	\documentauthor
	%\footnotesize \href{mailto:pablo@ppizarror.com}{pabflo@ppizarror.com}
}
\institute[UChile]{
	\includegraphics[height=1.1cm]{\universitydepartmentimage} \\
	\medskip
	\universityname \\
	\universityfaculty \\
	\universitydepartment
}
\date[\today]{\footnotesize{\today}}

% IMPORTACIÓN DEL TEMPLATE
\input{template}

% INICIO DE LAS PÁGINAS
\begin{document}

% CONFIGURACIÓN DE PÁGINA Y ENCABEZADOS
\templatePagecfg

% CONFIGURACIONES FINALES
\templateFinalcfg

% ======================= INICIO DEL DOCUMENTO =======================

% Portada
\begin{frame}
	\titlepage
\end{frame}

% Tabla de contenidos
%\begin{frame}
%	\frametitle{Contenidos}
%	\tableofcontents
%\end{frame}

% Organiza el documento en secciones, útiles para la tabla de contenidos
\section{Primera sección}

\subsection{Ejemplo de sub-sección}

%---------------------------------------------------------------------
% eagleNebula.jpeg

\begin{frame}{Mediciones de SFR mediante lineas}
	
	\insertimage[\label{img:SFR}]{starFormationEverywhere.jpg}{scale=0.042}{Zona de formación estelar HII. Creditos: NASA/JPL-Caltech/UCLA}
	
\end{frame}

%---------------------------------------------------------------------
% eagleNebula.jpeg

\begin{frame}{Efectos del polvo en la luz emitida por las estrellas.}
	
		\insertimage[\label{img:eagle}]{eagleNebula.jpeg}{scale=0.99}{Nebulosa del águila. Créditos: NASA, ESA, and the Hubble Heritage.}
	
\end{frame}

%---------------------------------------------------------------------
%starFormationEverywhere.jpg



\begin{frame}{Ventajas del estudio de análogos locales}
	
	Las lineas de emisión del infrarrojo cercano se escapan de las capacidades observacionales alrededor de $z∼2$.
	
	\insertimage[\label{img:JWST}]{JWST.png}{scale=0.27}{Distintos modos de observación de JWST. Créditos: NASA y STScI.}
	
	
\end{frame}

%---------------------------------------------------------------------

\begin{frame}{Seleccionando análogos locales}
	 %\footnote{Insertar footnotes es muy fácil con el template!}
	\insertimage[\label{img:BPT}]{BPT.png}{scale=0.2}{Diagrama BPT, fuente: Bian et al. 2016}
	
\end{frame}

%---------------------------------------------------------------------

\begin{frame}{Datos espectrograficos de magelanFIRE}
	% \addimage{fuente.png}{width=4cm}{Observación cruda de la fuente (J2215+0002).}
	\insertimage[\label{img:specRaw}]{dataZoom.png}{scale=0.25}{Observación cruda de la fuente (J2215+0002).}
\end{frame}


%---------------------------------------------------------------------

\begin{frame}{Datos espectrograficos de magelanFIRE}
	% \addimage{fuente.png}{width=4cm}{Observación cruda de la fuente (J2215+0002).}
	\insertimage[\label{img:spec}]{Detalle.png}{scale=0.25}{Detalle de observación cruda de la fuente (J2215+0002).}
\end{frame}

%---------------------------------------------------------------------

\begin{frame}{Pasos de la reducción de datos con Pypeit\footnote{Creditos de pypeit: Prochaska et al. 2020}}
	
	\begin{itemize}
		\item \textbf{Flat Fielding}
		\item \textbf{Trazado de ordenes}
		\item \textbf{Solución para longitudes de onda}
		\item \textbf{Extracción del cielo}
		\item \textbf{Calibración por flujo}
		\item \textbf{Se combinan las imágenes}
		\item \textbf{Corrección por telúrica}
	\end{itemize}
\end{frame}

%---------------------------------------------------------------------

\begin{frame}{Espectro reducido completo.}
	\insertimage[\label{img:specCom}]{espectroCompleto.png}{scale=0.27}{Espectro reducido de la galaxia.} \end{frame}

%---------------------------------------------------------------------

\begin{frame}{Close up del espectro}
	\insertimage[\label{img:spec1}]{zona1.png}{scale=0.27}{Espectro reducido de la galaxia.} \end{frame}

%---------------------------------------------------------------------

\begin{frame}{Paschen-$\alpha$}
	\insertimage[\label{img:PaAlpha}]{pa.pdf}{scale=0.55}{Linea Paschen alpha reducida y modelada con tres componentes cinemáticos.} \end{frame}

%---------------------------------------------------------------------

\begin{frame}{Paschen-$\alpha$}
	
	\begin{itemize}
	\item{\textbf{Redshift}}
	Comparando lineas de hidrógeno intrínsecas vs observadas se estimó un redshift de: $0.078$
	\item{\textbf{Flujo observado}}
		Integrando el área bajo las gaussianas se obtuvo un flujo de:\\ $(7.07 \pm 0.11)$ $\frac{10^{-15} \cdot ergs}{s \cdot cm^2}$

	
	\item{\textbf{Flujo intrínseco}}
		A partir del redshift, se estimó la distancia a la galaxia, con la que se obtuvo:\\ $(1.06 \pm 0.01)$ $\frac{10^{41} \cdot ergs}{s}$
		

	
	
	\item{\textbf{Taza de formación estelar:}}
		Usando el diagnostico de Kennicutt (1999), la taza de formación estelar se estima en:\\
		$6.56 \pm 0.10$ $\frac{M_{\odot}}{yr}$
	\end{itemize}
\end{frame}

\begin{frame}{Prospectos futuros.}
	
	\begin{itemize}
	\item{\textbf{Análisis bayesiano del modelo de tres componentes.}}

	\item{\textbf{Utilizar los flujos de distintas lineas para estudiar otras propiedades físicas de la galaxias.}}
	
	
	\item{\textbf{Modelar la extinción por polvo de la galaxia.}}

\end{itemize}

\end{frame}

%---------------------------------------------------------------------

\begin{frame}
	\centering
	\Huge{Conclusiones}
\end{frame}

%---------------------------------------------------------------------

\begin{frame}
	\centering
	\Huge{Gracias por su atención}
\end{frame}
% FIN DEL DOCUMENTO
\end{document}